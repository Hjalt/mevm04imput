\documentclass[headinclude=true, headsepline=true, DIV14]{scrartcl}\usepackage[]{graphicx}\usepackage[]{color}
%% maxwidth is the original width if it is less than linewidth
%% otherwise use linewidth (to make sure the graphics do not exceed the margin)
\makeatletter
\def\maxwidth{ %
  \ifdim\Gin@nat@width>\linewidth
    \linewidth
  \else
    \Gin@nat@width
  \fi
}
\makeatother

\definecolor{fgcolor}{rgb}{0.345, 0.345, 0.345}
\newcommand{\hlnum}[1]{\textcolor[rgb]{0.686,0.059,0.569}{#1}}%
\newcommand{\hlstr}[1]{\textcolor[rgb]{0.192,0.494,0.8}{#1}}%
\newcommand{\hlcom}[1]{\textcolor[rgb]{0.678,0.584,0.686}{\textit{#1}}}%
\newcommand{\hlopt}[1]{\textcolor[rgb]{0,0,0}{#1}}%
\newcommand{\hlstd}[1]{\textcolor[rgb]{0.345,0.345,0.345}{#1}}%
\newcommand{\hlkwa}[1]{\textcolor[rgb]{0.161,0.373,0.58}{\textbf{#1}}}%
\newcommand{\hlkwb}[1]{\textcolor[rgb]{0.69,0.353,0.396}{#1}}%
\newcommand{\hlkwc}[1]{\textcolor[rgb]{0.333,0.667,0.333}{#1}}%
\newcommand{\hlkwd}[1]{\textcolor[rgb]{0.737,0.353,0.396}{\textbf{#1}}}%

\usepackage{framed}
\makeatletter
\newenvironment{kframe}{%
 \def\at@end@of@kframe{}%
 \ifinner\ifhmode%
  \def\at@end@of@kframe{\end{minipage}}%
  \begin{minipage}{\columnwidth}%
 \fi\fi%
 \def\FrameCommand##1{\hskip\@totalleftmargin \hskip-\fboxsep
 \colorbox{shadecolor}{##1}\hskip-\fboxsep
     % There is no \\@totalrightmargin, so:
     \hskip-\linewidth \hskip-\@totalleftmargin \hskip\columnwidth}%
 \MakeFramed {\advance\hsize-\width
   \@totalleftmargin\z@ \linewidth\hsize
   \@setminipage}}%
 {\par\unskip\endMakeFramed%
 \at@end@of@kframe}
\makeatother

\definecolor{shadecolor}{rgb}{.97, .97, .97}
\definecolor{messagecolor}{rgb}{0, 0, 0}
\definecolor{warningcolor}{rgb}{1, 0, 1}
\definecolor{errorcolor}{rgb}{1, 0, 0}
\newenvironment{knitrout}{}{} % an empty environment to be redefined in TeX

\usepackage{alltt}
%\usepackage[swedish]{babel}



% Fonts
\usepackage[utf8]{inputenc}
\usepackage[full]{textcomp} % to get the right copyright, etc.
\usepackage{fbb} % so math uses tabular lining figures
\usepackage[scaled=.97]{cabin} % sans serif in style of Gill Sans
\usepackage[zerostyle=c,scaled=.95]{newtxtt}
\usepackage[T1]{fontenc}
\usepackage[garamondx,bigdelims]{newtxmath}
\usepackage[cal=boondoxo,bb=boondox,frak=boondox]{mathalfa}
\usepackage[italic]{mathastext}
\useosf

% Floats and figures
\usepackage{float}
\usepackage{caption}
%\usepackage{graphicx}

% Tables and lists
\usepackage{enumitem}
\usepackage{tabu}
\usepackage{booktabs}

% Title page
\usepackage{abstract}
\usepackage{authblk}

% Headers
\usepackage{scrlayer-scrpage}
\pagestyle{scrheadings}
\lohead[]{Larsson}
\rohead[]{MEVM04: Imputing missing data}

% Hyperref setup
\usepackage[dvipsnames,svgnames,x11names,hyperref]{xcolor}
\usepackage[hyphens]{url}
\usepackage{hyperref}
\hypersetup{linkcolor=DodgerBlue4,citecolor=DodgerBlue4,urlcolor=DodgerBlue4,colorlinks = true}

% tikz
\usepackage{tikz}
\usetikzlibrary{shapes,arrows,decorations.text}
\usepackage{standalone}

\title{Imputing missing data}
\subtitle{MEVM04}
\author{Johan Larsson}
%\author[1]{Johan Larsson\thanks{johanlarsson@outlook.com; corresponding author.}
%\affil[1]{}

%\renewcommand\Affilfont{\itshape\small}
%\renewcommand{\sectfont}{\rmfamily\bfseries}
\IfFileExists{upquote.sty}{\usepackage{upquote}}{}
\begin{document}


\maketitle

\section{Introduction}

Missing values are common in clinical research and can occur for many reasons.
Crudely, they can be separated into missing values on the predictor(s) ($x$, also
known as \emph{independent variables}) and on the outcome(s) ($y$, also
known as \emph{dependent variables}). 

In the first case, missing data might occur because a participant did not wish
to disclose information on their weight (here, a predictor); in the second case,
observations might be missing because the blood pressure gauge had malfunctioned
when the participant's blood pressure was measured. Whether a value is missing
on $x$ or $y$ affect our ability to 1) do something about it, and 2) the
consequences for our results. In this text, we will deal with each
instance of missing values in turn.

A common choice when data is missing is \emph{case deletion}: deleting each observation
for which a value is missing, which is also known as doing a complete cases
analysis. Many common statistical models, analysis of variance for instance,
requires complete cases to work properly, whilst some, such as tree-based models,
do not~\cite{Kuhn_2013}; such models, however, are beyond the scope of this
text. Whilst it is frequent, case deletion is generally bad practice since it
can lead to unpredictable bias~\cite{knol_unpredictable_2010} and inefficiency~\cite{carpenterjames_multiple_2013}.

A better alternative to case deletion is \emph{imputation}, which is when
you fill in (impute) missing values based on information
given by other variables. In its simplest form, missing data on a person's
weight might be substituted with the mean weight of the entire group (though as
we shall see there are much better options available).

This text is an assignment for a class in my master's programme: \emph{MEVN04},
wherein I have attempted to outline the basics of missing data with a focus on
how to handle missing valued on predictor variables using imputation. The text 
is based heavily on books by Steyerberg \cite{steyerberg_clinical_2009} and 
Harrell \cite{Harrell_2015}; as a final step, I have provided a worked-through 
example given in the latter book.

As a secondary objective, this text is an attempt of reproducible research using
\emph{knitR} for \emph{R} (R Core Team). All of the files used to produce this
text (including plots and statistical analyses) will therefore be provided
publicly on Github at
\href{https://github.com/Hjalt/mevn04imput}{https://github.com/Hjalt/mevn04imput}.

\section{Types of missing data}

Apart from whether predictor or outcome variables are missing, there are 
three types of missing data:

\begin{description}[style=multiline,leftmargin=8ex]
  \item[MCAR] Missing Completely At Random
  \item[MAR] Missing At Random
  \item[MNAR] Missing Not At Random
\end{description}

MCAR means that there is no pattern to why the data is missing; perhaps the 
scale had malfunctioned and weight data had to be discarded. When data is MAR
there is conversely a pattern to its absence, although that pattern is apparent
from the data -- younger people might for example have refused to be weighed more
frequently (and we recorded age). Lastly,
when data is MNAR there is a pattern that cannot be deciphered from
the available data. Perhaps people of lower income did not want to step up on
the scale but we did not measure income. MNAR is a particularly problematic form
since it is by definition impossible to understand if and how it might be
affecting the results.

\section{Missing data on independent variables}

The situation where data is missing on predictors deserves most
attention as it is both common and manageable, whilst being detrimental to our
results (if not taken care of).

In \autoref{fig:matrix} various scenarios of missing predictor values
have been simulated from a simple linear model \[y=x1+x2+\text{error}\]

Here, $x1$, $x2$, and the error term are generated from a normal distribution
with a mean of 0, and a standard deviation of 1, 1, and 0.1 respectively. Thus,
the true slope of the relationship is $r=1$, which can be seen from the first
panel. This example has been adapted and revised from the code excerpts from
Steyerberg \cite{steyerberg_clinical_2009}.

\begin{figure}[t]
\centering
\begin{knitrout}
\definecolor{shadecolor}{rgb}{0.969, 0.969, 0.969}\color{fgcolor}
\includegraphics[width=\maxwidth]{figure/matrix-1} 

\end{knitrout}
\caption{Different types of missing data and their effects on the correlation coefficient. $x1$ and $x2$ are normally distributed. The true correlation coefficients (r) are 1. Only when data is MAR on $y$ is the correlation coefficient noticeably erroneous. This example has been adapted from Steyerberg \cite{steyerberg_clinical_2009}}
\label{fig:matrix}
\end{figure}

When data is MCAR on $x1$, the foremost consequence is that we lose
statistical power. If for instance every other participant had one variable 
missing out of 5, this would mean that half of the sample was dropped from the
analysis, even though only 10\% of the data was missing. As we see on the
slope, however, data that is MCAR does not actually affect the estimate, simply
because it is random and not systematic missingness.

In the third panel, $x1$ is missing for lower levels of $x2$, which is a case of
MAR. Here, too, is the slope unbiased (but would not be if the relationship was
different for lower, compared with higher, values of $x2$.

In the fourth panel, $x1$ is missing for higher levels of $y$ -- also a case of
MAR. We can see here that the slope estimate is biased low ($r=0.73$ for $x1$
and $r=0.76$ for $x2$).

Finally, in the fifth panel, $x1$ is missing for higher values of $x1$ -- a 
pattern that cannot be deduced from the data. This is an instance of MNAR. 
The slope is unbiased in this instance, which is related to missingness of $x1$
being related to $x1$ itself. If there was a third variable in the mix,
to which $x1$ owed its missingness, the results would be different.

\section{Single imputation}

\subsection{Simple mean imputation}

A simple, but flawed, option is to replace missing values with the mean of the
variable, which is suboptimal because it fails to take advantage of patterns in
the data, i.e., correlations between variables that otherwise can improve the
accuracy of imputations. Moreover, it underestimates the variance (the
randomness) of the estimates, which leads to overconfidence in the estimates.

Better options exist in \emph{conditional mean imputation}, 
\emph{stochastic regression imputation}, and \emph{multiple imputation}.

\subsection{Conditional mean imputation}

In conditional mean imputation (or regression imputation), missing data on a
variable are imputed using the patters of how the variable correlates with
other variables. As with simple mean imputation, however, conditional mean
imputation underestimates the variance of the imputations; if, for example,
we were to impute missing values in \autoref{fig:matrix} with conditional
mean imputation, all those values would lie along the imputation slope and this
would not reflect the true nature of variance.

\subsection{Stochastic regression imputation}

Stochastic regression imputation alleviates the issue of underestimated variance
by randomly sampling from the distribution of predicted values, that is, the
distribution that in conditional mean imputation would have been used to 
compute the mean, whereby it simulates the random error that is inherent to
all real measurements.

\section{Multiple imputation}

Though single imputation techniques are efficient and sometimes adequate, they 
present one drawback in that the randomness inherent to
the imputation step itself is not explicitly covered in the analysis. The 
simplest way of looking at this is to consider that only after we are done
imputing do we conduct our analysis \ref{fig:simput}, which means that any
bias we introduce with our imputation model is disregarded -- this poses a
serious problem for the external validity of our model.

\begin{figure}
\tikzstyle{sphere} = [circle, text badly centered, inner sep=1ex, text width = 15ex, draw]
\tikzstyle{empty} = [rectangle, inner sep=3ex]
\tikzstyle{line} = [draw, -latex']

\centering
\small
\begin{tikzpicture}[node distance=3cm, auto]
\node[empty](analysis){Analysis};

\node[sphere, below of = analysis](impset){Imputed data set};
\node[empty, left of = impset](emptynode){};
\node[sphere, left of = emptynode](firstset){Data set with missing values};

\path[line, dashed](analysis) -- (impset);
\path[line](firstset) -- node [] {Imputation} (impset);

\end{tikzpicture}
\caption{Flow chart of single imputation models.}
\label{fig:simput}
\end{figure}

This issue is solved with \emph{multiple imputation}, a statistical technique
that was designed by Rubin \cite{rubin_multiple_2004} as a flexible
method for dealing with missing data. In multiple imputation, several data sets
are produced by random sampling from the original data set. In each of these
sets an imputation model is fit (as in conditional mean imputation) and a
estimate is established as an aggregate of all of these data sets. Because we
are then including the variance of imputation in the aggregate, we directly
take bias from the imputation step into account.

\subsection{Chained equations}

If data is missing on several variables, it might be necessary to impute in
steps, which is known as chained equations multiple imputation. It works by
first imputing data on one variable, then creating a new sample from the
original data plus the imputations, and thereafter repeating the imputation
step. This iterative process can proceed for as long as desired, but most
software use a stopping-rule to drop out of the loop at some preestablished
point.

\begin{figure}[t]
\centering
\documentclass{standalone}

\usepackage[utf8]{inputenc}
\usepackage[full]{textcomp} % to get the right copyright, etc.
\usepackage{fbb} % so math uses tabular lining figures
\usepackage[scaled=.97]{cabin} % sans serif in style of Gill Sans
\usepackage[zerostyle=c]{newtxtt}
\usepackage[T1]{fontenc}
\usepackage[garamondx,bigdelims]{newtxmath}
\usepackage[cal=boondoxo,bb=boondox,frak=boondox]{mathalfa}
\usepackage[italic]{mathastext}
\useosf

\usepackage{tikz}
\usetikzlibrary{shapes,arrows,decorations.text}

\begin{document}
\tikzstyle{decision} = [circle, draw,text width = 8ex]%
\tikzstyle{square} = [rectangle, draw, inner sep = 1ex]%
\tikzstyle{line} = [draw, -stealth']%
\begin{tikzpicture}[text badly centered, node distance = 25ex, auto]
\footnotesize
\node[decision](start){Original dataset};
\node[decision](set1) at (54:25ex){Set 1};
\node[decision](set2) at (18:25ex){Set 2};
\node[decision](set3)  at (342:25ex) {Set 3};
\node[decision](setn)  at (306:25ex) {$\dots n$};
\node[square, right of = start, node distance = 75ex,text width=12ex](analysis){Averaged estimate};

\node[decision, right of=set1](set1imp){};
\node[decision, right of=set2](set2imp){};
\node[decision, right of=set3](set3imp){};
\node[decision, right of=setn](setnimp){};

\path[line](start) -- node [sloped,above]{\tiny{Random sampling}} (set1);
\path[line](start) -- (set2);
\path[line](start) -- (set3);
\path[line](start) -- (setn);

\draw[line](set1) -- node{\tiny{Imputation}} (set1imp);
\path[line](set2) -- (set2imp);
\path[line](set3) -- (set3imp);
\path[line](setn) -- (setnimp);

\def\myshift#1{\raisebox{1ex}}
\draw[-latex' reversed,dashed,postaction={decorate,decoration={text along path,text align=center,text={|\tiny\myshift|New imputation}}}] (set1.north) to [bend left=90] node{}(set1imp.north);
\def\myshift#1{\raisebox{1ex}}

\draw[line] (set1imp.east) to [bend left] node{}(analysis.north);
\draw[line] (set2imp.east) to [out=0, in=180] node{} (analysis.west);
\draw[line] (set3imp.east) to [out=0, in=180] node{}(analysis.west);
\draw[line] (setnimp.east) to [out=0, in=270] node{} (analysis.south); 

\end{tikzpicture}
\end{document}

\caption{Flow chart of multiple imputation model with chained equations. The
entire situation is only visualized for \emph{Set 1}.}
\label{fig:mimput}
\end{figure}

\section{Missing data on dependent variable}

Observations with missing data on the dependent variable (outcome) are commonly
excluded from the analysis. And if they are MCAR, there is an argument to do so
since no additional information can be gathered from the remaining data if
the variable of interest is missing \cite{Harrell_2015}.

Yet, when data on $Y$ is MAR, bias might occur. Harrell recommends that
researchers should as a minimum characterize any patterns of missingness, and
simulation studies show that there are benefits to imputation also on $y$~\cite{hippel_4_2007}.

\section{Issues with imputation}

In imputation we assume that data is MAR, i.e. we are able to figure out from
the rest of the variables why data is missing. However, this assumption is by
definition not testable \cite{grahamjohnw._missing_2012}
(but becomes increasingly tenable as more variables are measured).

If the assumption does not hold and data is instead MNAR then
imputation can only lead to increased bias, and it would be advisable to stick
to case deletion. However, as many empirical as well
as simulation experiments have shown, imputation generally leads to less bias
and more efficiency with relatively little bias.

\section{A worked-through example}




\subsection{Explore the variables and patterns of missing data in the SUPPORT
dataset.}
\subsubsection{Print univariate summaries of all variables. Make a plot (showing all
variables on one page) that describes especially the continuous variables.}

We start with some descriptive variables

\begin{knitrout}
\definecolor{shadecolor}{rgb}{0.969, 0.969, 0.969}\color{fgcolor}\begin{kframe}
\begin{alltt}
\hlkwd{library}\hlstd{(Hmisc)}

\hlcom{# Load the SUPPORT data set.}
\hlkwd{getHdata}\hlstd{(support)}

\hlcom{# Total proportion of missing values}
\hlkwd{sum}\hlstd{(}\hlkwd{sapply}\hlstd{(support, is.na))}\hlopt{/}\hlkwd{length}\hlstd{(}\hlkwd{sapply}\hlstd{(support, is.na))}
\end{alltt}
\begin{verbatim}
## [1] 0.1375429
\end{verbatim}
\begin{alltt}
\hlcom{# Print some summary statistics of every variable}
\hlkwd{describe}\hlstd{(support)}
\end{alltt}
\begin{verbatim}
## support 
## 
##  35  Variables      1000  Observations
## -----------------------------------------------------------------
## age : Age 
##       n missing  unique    Info    Mean     .05     .10     .25 
##    1000       0     970       1   62.47   33.76   38.91   51.81 
##     .50     .75     .90     .95 
##   64.90   74.50   81.87   86.00 
## 
## lowest :  18.04  18.41  19.76  20.30  20.31
## highest:  95.51  96.02  96.71 100.13 101.85 
## -----------------------------------------------------------------
## death : Death at any time up to NDI date:31DEC94 
##       n missing  unique    Info     Sum    Mean 
##    1000       0       2    0.67     668   0.668 
## -----------------------------------------------------------------
## sex 
##       n missing  unique 
##    1000       0       2 
## 
## female (438, 44%), male (562, 56%) 
## -----------------------------------------------------------------
## hospdead : Death in Hospital 
##       n missing  unique    Info     Sum    Mean 
##    1000       0       2    0.57     253   0.253 
## -----------------------------------------------------------------
## slos : Days from Study Entry to Discharge 
##       n missing  unique    Info    Mean     .05     .10     .25 
##    1000       0      88       1   17.86       4       4       6 
##     .50     .75     .90     .95 
##      11      20      37      53 
## 
## lowest :   3   4   5   6   7, highest: 145 164 202 236 241 
## -----------------------------------------------------------------
## d.time : Days of Follow-Up 
##       n missing  unique    Info    Mean     .05     .10     .25 
##    1000       0     582       1   475.7     5.0     8.0    27.0 
##     .50     .75     .90     .95 
##   256.5   725.0  1464.3  1757.1 
## 
## lowest :    3    4    5    6    7
## highest: 2006 2011 2022 2024 2029 
## -----------------------------------------------------------------
## dzgroup 
##       n missing  unique 
##    1000       0       8 
## 
##           ARF/MOSF w/Sepsis COPD CHF Cirrhosis Coma Colon Cancer
## Frequency               391  116 143        55   60           49
## %                        39   12  14         6    6            5
##           Lung Cancer MOSF w/Malig
## Frequency         100           86
## %                  10            9
## -----------------------------------------------------------------
## dzclass 
##       n missing  unique 
##    1000       0       4 
## 
## ARF/MOSF (477, 48%) 
## COPD/CHF/Cirrhosis (314, 31%) 
## Coma (60, 6%), Cancer (149, 15%) 
## -----------------------------------------------------------------
## num.co : number of comorbidities 
##       n missing  unique    Info    Mean 
##    1000       0       8    0.94   1.886 
## 
##             0   1   2   3  4  5  6 7
## Frequency 120 337 269 151 76 29 17 1
## %          12  34  27  15  8  3  2 0
## -----------------------------------------------------------------
## edu : Years of Education 
##       n missing  unique    Info    Mean     .05     .10     .25 
##     798     202      25    0.97   11.78       6       8      10 
##     .50     .75     .90     .95 
##      12      14      16      18 
## 
## lowest :  0  1  2  3  4, highest: 20 21 22 24 30 
## -----------------------------------------------------------------
## income 
##       n missing  unique 
##     651     349       4 
## 
## under $11k (309, 47%), $11-$25k (161, 25%) 
## $25-$50k (106, 16%), >$50k (75, 12%) 
## -----------------------------------------------------------------
## scoma : SUPPORT Coma Score based on Glasgow D3 
##       n missing  unique    Info    Mean     .05     .10     .25 
##    1000       0      11    0.65   11.74     0.0     0.0     0.0 
##     .50     .75     .90     .95 
##     0.0     9.0    44.0    62.4 
## 
##             0  9 26 37 41 44 55 61 89 94 100
## Frequency 704 83 58 24 19 45 11  6  8  7  35
## %          70  8  6  2  2  4  1  1  1  1   4
## -----------------------------------------------------------------
## charges : Hospital Charges 
##       n missing  unique    Info    Mean     .05     .10     .25 
##     975      25     967       1   56271    3757    4688   10029 
##     .50     .75     .90     .95 
##   26499   63622  147109  223582 
## 
## lowest :   1636   1680   1830   2045   2082
## highest: 504660 538323 543761 706577 740010 
## -----------------------------------------------------------------
## totcst : Total RCC cost 
##       n missing  unique    Info    Mean     .05     .10     .25 
##     895     105     895       1   30490    2484    3081    5899 
##     .50     .75     .90     .95 
##   15110   37598   72906  114932 
## 
## lowest :      0   1162   1201   1285   1396
## highest: 269131 299966 338955 357919 390461 
## -----------------------------------------------------------------
## totmcst : Total micro-cost 
##       n missing  unique    Info    Mean     .05     .10     .25 
##     628     372     617       1   26168    1653    2548    5297 
##     .50     .75     .90     .95 
##   13828   33691   66229   96753 
## 
## lowest :      0.0    829.6    834.6    914.4   1016.4
## highest: 154709.0 198047.0 234875.9 246904.0 271467.2 
## -----------------------------------------------------------------
## avtisst : Average TISS, Days 3-25 
##       n missing  unique    Info    Mean     .05     .10     .25 
##     994       6     241       1   22.64    6.00    8.00   12.00 
##     .50     .75     .90     .95 
##   19.00   31.75   43.33   48.00 
## 
## lowest :  1.667  2.500  3.000  3.500  4.000
## highest: 58.500 59.000 60.000 61.000 64.000 
## -----------------------------------------------------------------
## race 
##       n missing  unique 
##     995       5       5 
## 
##           white black asian other hispanic
## Frequency   781   157     9    12       36
## %            78    16     1     1        4
## -----------------------------------------------------------------
## meanbp : Mean Arterial Blood Pressure Day 3 
##       n missing  unique    Info    Mean     .05     .10     .25 
##    1000       0     122       1   84.98   47.00   55.00   64.75 
##     .50     .75     .90     .95 
##   78.00  107.00  120.00  128.05 
## 
## lowest :   0  20  27  30  32, highest: 155 158 161 162 180 
## -----------------------------------------------------------------
## wblc : White Blood Cell Count Day 3 
##       n missing  unique    Info    Mean     .05     .10     .25 
##     976      24     282       1    12.4   2.475   4.800   6.899 
##     .50     .75     .90     .95 
##  10.449  15.500  22.248  27.524 
## 
## lowest :   0.05000   0.06999   0.09999   0.14999   0.19998
## highest:  60.00000  61.19531  67.09375  79.39062 100.00000 
## -----------------------------------------------------------------
## hrt : Heart Rate Day 3 
##       n missing  unique    Info    Mean     .05     .10     .25 
##    1000       0     124       1   97.87    54.0    60.0    72.0 
##     .50     .75     .90     .95 
##   100.0   120.0   135.0   146.1 
## 
## lowest :   0  11  30  35  36, highest: 189 193 199 232 300 
## -----------------------------------------------------------------
## resp : Respiration Rate Day 3 
##       n missing  unique    Info    Mean     .05     .10     .25 
##    1000       0      45    0.99   23.49       9      10      18 
##     .50     .75     .90     .95 
##      24      29      36      40 
## 
## lowest :  0  4  6  7  8, highest: 48 49 52 60 64 
## -----------------------------------------------------------------
## temp : Temperature (celcius) Day 3 
##       n missing  unique    Info    Mean     .05     .10     .25 
##    1000       0      64       1   37.08   35.50   35.80   36.20 
##     .50     .75     .90     .95 
##   36.70   38.09   38.80   39.20 
## 
## lowest : 32.50 33.70 34.00 34.09 34.40
## highest: 40.20 40.59 40.90 41.00 41.20 
## -----------------------------------------------------------------
## pafi : PaO2/(.01*FiO2) Day 3 
##       n missing  unique    Info    Mean     .05     .10     .25 
##     747     253     463       1   244.2   92.61  115.00  156.33 
##     .50     .75     .90     .95 
##  226.66  310.00  400.00  442.81 
## 
## lowest :  34.00  39.00  45.00  48.00  53.33
## highest: 600.00 623.75 640.00 680.00 869.38 
## -----------------------------------------------------------------
## alb : Serum Albumin Day 3 
##       n missing  unique    Info    Mean     .05     .10     .25 
##     622     378      38       1   2.917   1.800   2.000   2.400 
##     .50     .75     .90     .95 
##   2.800   3.400   4.000   4.199 
## 
## lowest : 1.100 1.200 1.300 1.400 1.500
## highest: 4.500 4.600 4.699 4.800 4.899 
## -----------------------------------------------------------------
## bili : Bilirubin Day 3 
##       n missing  unique    Info    Mean     .05     .10     .25 
##     703     297     115       1   2.527  0.3000  0.3000  0.5000 
##     .50     .75     .90     .95 
##  0.7999  1.7998  5.8594 12.5896 
## 
## lowest :  0.09999  0.19998  0.29999  0.39996  0.50000
## highest: 32.39844 33.00000 35.00000 39.29688 50.09375 
## -----------------------------------------------------------------
## crea : Serum creatinine Day 3 
##       n missing  unique    Info    Mean     .05     .10     .25 
##     997       3      87       1   1.808  0.6000  0.7000  0.8999 
##     .50     .75     .90     .95 
##  1.2000  1.8999  3.6396  5.5996 
## 
## lowest :  0.3  0.4  0.5  0.6  0.7
## highest: 10.4 10.6 11.2 11.6 11.8 
## -----------------------------------------------------------------
## sod : Serum sodium Day 3 
##       n missing  unique    Info    Mean     .05     .10     .25 
##    1000       0      42       1   137.7     129     131     134 
##     .50     .75     .90     .95 
##     137     141     145     148 
## 
## lowest : 118 120 121 122 123, highest: 156 157 158 168 175 
## -----------------------------------------------------------------
## ph : Serum pH (arterial) Day 3 
##       n missing  unique    Info    Mean     .05     .10     .25 
##     750     250      53       1   7.416   7.289   7.319   7.380 
##     .50     .75     .90     .95 
##   7.420   7.470   7.500   7.520 
## 
## lowest : 6.960 6.989 7.069 7.119 7.130
## highest: 7.590 7.600 7.609 7.659 7.670 
## -----------------------------------------------------------------
## glucose : Glucose Day 3 
##       n missing  unique    Info    Mean     .05     .10     .25 
##     530     470     226       1   156.4    74.0    82.0   100.0 
##     .50     .75     .90     .95 
##   128.0   185.0   269.3   327.5 
## 
## lowest :  11  25  30  42  51, highest: 492 512 528 576 598 
## -----------------------------------------------------------------
## bun : BUN Day 3 
##       n missing  unique    Info    Mean     .05     .10     .25 
##     545     455     106       1   32.61     7.0     9.0    14.0 
##     .50     .75     .90     .95 
##    23.0    43.0    68.6    88.8 
## 
## lowest :   1   2   3   4   5, highest: 124 125 127 128 146 
## -----------------------------------------------------------------
## urine : Urine Output Day 3 
##       n missing  unique    Info    Mean     .05     .10     .25 
##     483     517     359       1    2194   141.7   600.0  1208.5 
##     .50     .75     .90     .95 
##  1925.0  2900.0  4087.6  4822.5 
## 
## lowest :    0    1    5    8   15
## highest: 7275 7360 7455 7560 7750 
## -----------------------------------------------------------------
## adlp : ADL Patient Day 3 
##       n missing  unique    Info    Mean 
##     366     634       8    0.84   1.246 
## 
##             0  1  2  3  4  5  6 7
## Frequency 194 73 28 20 12 19 16 4
## %          53 20  8  5  3  5  4 1
## -----------------------------------------------------------------
## adls : ADL Surrogate Day 3 
##       n missing  unique    Info    Mean 
##     690     310       8     0.9   1.755 
## 
##             0   1  2  3  4  5  6  7
## Frequency 313 126 54 39 42 34 52 30
## %          45  18  8  6  6  5  8  4
## -----------------------------------------------------------------
## sfdm2 
##       n missing  unique 
##     841     159       5 
## 
## no(M2 and SIP pres) (326, 39%) 
## adl>=4 (>=5 if sur) (111, 13%) 
## SIP>=30 (59, 7%), Coma or Intub (7, 1%) 
## <2 mo. follow-up (338, 40%) 
## -----------------------------------------------------------------
## adlsc : Imputed ADL Calibrated to Surrogate 
##       n missing  unique    Info    Mean     .05     .10     .25 
##    1000       0     251    0.97    1.98   0.000   0.000   0.000 
##     .50     .75     .90     .95 
##   1.670   3.042   5.000   6.000 
## 
## lowest : 0.0000 0.4948 0.4948 1.0000 1.1667
## highest: 5.9932 6.0000 6.3398 6.4658 7.0000 
## -----------------------------------------------------------------
\end{verbatim}
\end{kframe}
\end{knitrout}

To make a plot, we first drop the categorical variables from the data set, then
we compute a correlation matrix, reorganize it with a factor
analysis and then plot it using the facilities of \texttt{levelplot} in
lattice.

\begin{knitrout}
\definecolor{shadecolor}{rgb}{0.969, 0.969, 0.969}\color{fgcolor}\begin{kframe}
\begin{alltt}
\hlcom{# Subset the support data set to remove categorical variables}
\hlstd{support.cont} \hlkwb{<-} \hlstd{support[,} \hlkwd{sapply}\hlstd{(support, is.numeric)]}
\hlstd{support.cont} \hlkwb{<-} \hlstd{support.cont[,} \hlopt{-}\hlstd{(}\hlnum{2}\hlopt{:}\hlnum{3}\hlstd{)]}

\hlcom{# Print a scatterplot matrix}
\hlkwd{library}\hlstd{(psych)}
\hlkwd{library}\hlstd{(lattice)}
\hlkwd{library}\hlstd{(latticeExtra)}

\hlstd{m} \hlkwb{<-} \hlkwd{mat.sort}\hlstd{(}\hlkwd{as.matrix}\hlstd{(}\hlkwd{cor}\hlstd{(support.cont,}
                            \hlkwc{use} \hlstd{=} \hlstr{"pairwise.complete.obs"}\hlstd{)))}

\hlkwd{levelplot}\hlstd{(m,}
          \hlkwc{scales} \hlstd{=} \hlkwd{list}\hlstd{(}\hlkwc{x} \hlstd{=} \hlkwd{list}\hlstd{(}\hlkwc{rot} \hlstd{=} \hlnum{90}\hlstd{)),}
          \hlkwc{at} \hlstd{=} \hlkwd{seq}\hlstd{(}\hlopt{-}\hlnum{1}\hlstd{,} \hlnum{1}\hlstd{,} \hlkwc{length.out} \hlstd{=} \hlnum{100}\hlstd{))}
\end{alltt}
\end{kframe}

{\centering \includegraphics[width=\maxwidth]{figure/unnamed-chunk-4-1} 

}



\end{knitrout}

\subsection{Make a plot showing the extent of missing data and tendencies for some
variables to be missing on the same patients.}

We use Harrel's \texttt{Hmisc} package to plot the relationships of
missingness between variables.

\begin{knitrout}
\definecolor{shadecolor}{rgb}{0.969, 0.969, 0.969}\color{fgcolor}\begin{kframe}
\begin{alltt}
\hlkwd{library}\hlstd{(Hmisc)}
\hlkwd{naplot}\hlstd{(}\hlkwd{naclus}\hlstd{(support),} \hlkwc{which} \hlstd{=} \hlstr{"all"}\hlstd{)}
\end{alltt}
\end{kframe}

{\centering \includegraphics[width=\maxwidth]{figure/unnamed-chunk-5-1} 

}




{\centering \includegraphics[width=\maxwidth]{figure/unnamed-chunk-5-2} 

}




{\centering \includegraphics[width=\maxwidth]{figure/unnamed-chunk-5-3} 

}




{\centering \includegraphics[width=\maxwidth]{figure/unnamed-chunk-5-4} 

}



\end{knitrout}

\subsection{Total hospital costs (variable totcst) were estimated from 
hospitalspecific Medicare cost-to-charge ratios. Characterize what kind of
patients have missing totcst. For this characterization use the following
patient descriptors: age, sex, dzgroup, num.co, edu, income, scoma,
meanbp, hrt, resp, temp.}

We can solve this by plotting distributions of the various variables involved.

\begin{knitrout}
\definecolor{shadecolor}{rgb}{0.969, 0.969, 0.969}\color{fgcolor}\begin{kframe}
\begin{alltt}
\hlkwd{library}\hlstd{(dplyr)} \hlcom{# For data wrangling}
\hlstd{support.na} \hlkwb{<-} \hlstd{support[}\hlkwd{is.na}\hlstd{(support}\hlopt{$}\hlstd{totcst), ]} \hlopt
 \hlkwd{select}\hlstd{(age, sex, dzgroup, num.co, edu,}
        \hlstd{income, scoma, meanbp, hrt, resp, temp)}
\hlkwd{trellis.par.set}\hlstd{(}\hlkwc{fontsize} \hlstd{=} \hlkwd{list}\hlstd{(}\hlkwc{text} \hlstd{=} \hlnum{8}\hlstd{,} \hlkwc{points} \hlstd{=} \hlnum{6}\hlstd{))}
\hlkwd{marginal.plot}\hlstd{(support.na)}
\end{alltt}
\end{kframe}

{\centering \includegraphics[width=\maxwidth]{figure/unnamed-chunk-6-1} 

}



\end{knitrout}

\subsection{Prepare for later development of a model to predict costs by developing
reliable imputations for missing costs. Remove the observation having zero
totcst.}

We start by removing values for which the outcome (\texttt{totcst}) is missing
or zero.

\begin{knitrout}
\definecolor{shadecolor}{rgb}{0.969, 0.969, 0.969}\color{fgcolor}\begin{kframe}
\begin{alltt}
\hlstd{support.complete} \hlkwb{<-} \hlkwd{subset}\hlstd{(support,} \hlopt{!}\hlkwd{is.na}\hlstd{(totcst)} \hlopt{&} \hlstd{totcst} \hlopt{>} \hlnum{0}\hlstd{)}
\end{alltt}
\end{kframe}
\end{knitrout}

\subsubsection{The cost estimates are not available on 105 patients. Total hospital
charges (bills) are available on all but 25 patients. Relate these two
variables to each other with an eye toward using charges to predict
totcst when totcst is missing. Make graphs that will tell whether linear
regression or linear regression after taking logs of both variables is
better.}

\begin{knitrout}
\definecolor{shadecolor}{rgb}{0.969, 0.969, 0.969}\color{fgcolor}\begin{kframe}
\begin{alltt}
\hlcom{# As suggested, we use a log transformation}
\hlstd{support.complete}\hlopt{$}\hlstd{totcst.log} \hlkwb{<-} \hlkwd{log}\hlstd{(support.complete}\hlopt{$}\hlstd{totcst)}
\hlstd{support.complete}\hlopt{$}\hlstd{charges.log} \hlkwb{<-} \hlkwd{log}\hlstd{(support.complete}\hlopt{$}\hlstd{charges)}

\hlkwd{xyplot}\hlstd{(totcst.log} \hlopt{~} \hlstd{charges.log, support.complete,}
       \hlkwc{panel} \hlstd{=} \hlkwa{function}\hlstd{(}\hlkwc{...}\hlstd{) \{}
         \hlkwd{panel.xyplot}\hlstd{(...)}
         \hlkwd{panel.smoother}\hlstd{(...,} \hlkwc{span} \hlstd{=} \hlnum{0.9}\hlstd{)}
\hlstd{\})}
\end{alltt}
\end{kframe}

{\centering \includegraphics[width=\maxwidth]{figure/unnamed-chunk-8-1} 

}



\end{knitrout}

\subsubsection{Impute missing total hospital costs in SUPPORT based on a regression
model relating charges to costs, when charges are available.}

We need to specify an imputation model, which means we need to find the slope
and intercept of the regression model

\begin{knitrout}
\definecolor{shadecolor}{rgb}{0.969, 0.969, 0.969}\color{fgcolor}\begin{kframe}
\begin{alltt}
\hlstd{support.lm} \hlkwb{<-} \hlkwd{lm}\hlstd{(totcst.log} \hlopt{~} \hlstd{charges.log, support.complete)}
\hlstd{i} \hlkwb{<-} \hlkwd{coefficients}\hlstd{(support.lm)[}\hlnum{1}\hlstd{]} \hlcom{# intercept}
\hlstd{b} \hlkwb{<-} \hlkwd{coefficients}\hlstd{(support.lm)[}\hlnum{2}\hlstd{]} \hlcom{# slope}

\hlcom{# Impute values}
\hlstd{support.tf} \hlkwb{<-} \hlkwd{transform}\hlstd{(support,}
                        \hlkwc{totcst} \hlstd{=} \hlkwd{ifelse}\hlstd{(}\hlkwd{is.na}\hlstd{(totcst),}
                                 \hlstd{(}\hlkwd{exp}\hlstd{(i} \hlopt{+} \hlstd{b} \hlopt{*} \hlkwd{log}\hlstd{(charges))),}
                                 \hlstd{totcst))}
\end{alltt}
\end{kframe}
\end{knitrout}

\subsubsection{Compute the likely error in approximating total cost using charges by
computing the median absolute difference between predicted and observed
total costs in the patients having both variables available. If you
used a log transformation, also compute the median absolute percent
error in imputing total costs by anti-logging the absolute difference in
predicted logs.}

We start in reverse with the mmedian absolute percent error:

\begin{knitrout}
\definecolor{shadecolor}{rgb}{0.969, 0.969, 0.969}\color{fgcolor}\begin{kframe}
\begin{alltt}
\hlstd{a} \hlkwb{<-} \hlstd{support.complete[}\hlkwd{complete.cases}\hlstd{(support.complete}\hlopt{$}\hlstd{charges),}
                      \hlstr{"totcst.log"}\hlstd{]}
\hlstd{b} \hlkwb{<-} \hlkwd{predict}\hlstd{(support.lm)}
\hlkwd{exp}\hlstd{(}\hlkwd{mad}\hlstd{(a} \hlopt{-} \hlstd{b))} \hlcom{# The exponential of the natural logarithm}
\end{alltt}
\begin{verbatim}
## Total RCC cost 
## [1] 1.179605
\end{verbatim}
\end{kframe}
\end{knitrout}

Next, the median absolute deviation:

\begin{knitrout}
\definecolor{shadecolor}{rgb}{0.969, 0.969, 0.969}\color{fgcolor}\begin{kframe}
\begin{alltt}
\hlstd{a} \hlkwb{<-} \hlstd{support.complete[}\hlkwd{complete.cases}\hlstd{(support.complete}\hlopt{$}\hlstd{charges),}
                      \hlstr{"totcst"}\hlstd{]}
\hlstd{b} \hlkwb{<-} \hlkwd{exp}\hlstd{(}\hlkwd{predict}\hlstd{(support.lm))}
\hlkwd{mad}\hlstd{(a} \hlopt{-} \hlstd{b)} \hlcom{# In absolutes}
\end{alltt}
\begin{verbatim}
## Total RCC cost 
## [1] 2652.82
\end{verbatim}
\end{kframe}
\end{knitrout}

\subsection{State briefly why single conditional median imputation 
is OK here.}

\begin{knitrout}
\definecolor{shadecolor}{rgb}{0.969, 0.969, 0.969}\color{fgcolor}\begin{kframe}
\begin{alltt}
\hlstd{lm.cc} \hlkwb{<-} \hlkwd{lm}\hlstd{(totcst} \hlopt{~} \hlstd{charges, support.complete)}
\hlstd{lm.im} \hlkwb{<-} \hlkwd{lm}\hlstd{(totcst} \hlopt{~} \hlstd{charges, support.tf)}
\end{alltt}
\end{kframe}
\end{knitrout}

Although we have quite a large proportion of missing values (14\%), we are
introducing very little bias since our median absolute percent error is only 
1.18\%. Simply put: because all observed outcomes lie close to the
predicted ones, the bias we get from assuming that our missing values are on
the slope is neglible.

\subsection{Use transcan to develop single imputations for total cost, commenting on
the strength of the model fitted by transcan as well as how strongly each
variable can be predicted from all the others.}

I decided to pick only some of the variables.

\begin{knitrout}
\definecolor{shadecolor}{rgb}{0.969, 0.969, 0.969}\color{fgcolor}\begin{kframe}
\begin{alltt}
\hlstd{support.trans} \hlkwb{<-} \hlkwd{transcan}\hlstd{(}\hlopt{~} \hlstd{sex} \hlopt{+} \hlstd{age} \hlopt{+} \hlstd{totcst} \hlopt{+} \hlstd{death,} \hlkwc{data} \hlstd{= support,}
                          \hlkwc{imputed} \hlstd{= T,} \hlkwc{transformed}\hlstd{=}\hlnum{TRUE}\hlstd{)}
\end{alltt}
\end{kframe}

{\centering \includegraphics[width=\maxwidth]{figure/unnamed-chunk-13-1} 

}




{\centering \includegraphics[width=\maxwidth]{figure/unnamed-chunk-13-2} 

}




{\centering \includegraphics[width=\maxwidth]{figure/unnamed-chunk-13-3} 

}




{\centering \includegraphics[width=\maxwidth]{figure/unnamed-chunk-13-4} 

}



\end{knitrout}
\begin{knitrout}
\definecolor{shadecolor}{rgb}{0.969, 0.969, 0.969}\color{fgcolor}\begin{kframe}
\begin{alltt}
\hlkwd{summary}\hlstd{(support.trans)}
\end{alltt}
\begin{verbatim}
## transcan(x = ~sex + age + totcst + death, imputed = T, transformed = TRUE, 
##     data = support)
## 
## Iterations: 5 
## 
## R-squared achieved in predicting each variable:
## 
##    sex    age totcst  death 
##  0.019  0.113  0.058  0.022 
## 
## Adjusted R-squared:
## 
##    sex    age totcst  death 
##  0.014  0.106  0.050  0.017 
## 
## Coefficients of canonical variates for predicting each (row) variable
## 
##        sex   age   totcst death
## sex           0.70  0.83  -0.95
## age     0.26       -0.87   0.84
## totcst  0.37 -1.04         0.35
## death  -0.03  0.07  0.02       
## 
## Summary of imputed values
## 
## totcst 
##       n missing  unique    Info    Mean     .05     .10     .25 
##     105       0      99       1   55123    5853    7291    8965 
##     .50     .75     .90     .95 
##   14743   28658  210999  390460 
## 
## lowest :   1528   1719   2213   4376   4595
## highest: 238595 313517 313661 375291 390460 
## 
## Starting estimates for imputed values:
## 
##        sex        age     totcst      death 
##     2.0000    64.8965 15110.1000     1.0000
\end{verbatim}
\end{kframe}
\end{knitrout}

\texttt{totcst} can be imputed beneficially by \texttt{sex} and \texttt{age} but
not by \texttt{death}.

\subsection{Use predictive mean matching to multiply impute cost 10 times per missing
observation. Describe graphically the distributions of imputed values and
briefly compare these to distributions of non-imputed values. State in a simple 
way what the sample variance of multiple imputations for a single
observation of a continuous predictor is approximating.}

We use the \texttt{aregImpute} function here.

\begin{knitrout}
\definecolor{shadecolor}{rgb}{0.969, 0.969, 0.969}\color{fgcolor}\begin{kframe}
\begin{alltt}
\hlstd{support.aregImp} \hlkwb{<-} \hlkwd{aregImpute}\hlstd{(}\hlopt{~} \hlstd{age} \hlopt{+} \hlstd{sex} \hlopt{+} \hlstd{slos} \hlopt{+}
                                \hlstd{num.co} \hlopt{+} \hlstd{totcst} \hlopt{+} \hlstd{charges,}
                              \hlkwc{data} \hlstd{= support,}
                              \hlkwc{type} \hlstd{=} \hlstr{"pmm"}\hlstd{,}
                              \hlkwc{n.impute} \hlstd{=} \hlnum{10}\hlstd{)}
\end{alltt}
\begin{verbatim}
## Iteration 1 
Iteration 2 
Iteration 3 
Iteration 4 
Iteration 5 
Iteration 6 
Iteration 7 
Iteration 8 
Iteration 9 
Iteration 10 
Iteration 11 
Iteration 12 
Iteration 13 

\end{verbatim}
\begin{alltt}
\hlstd{support.aregImp}
\end{alltt}
\begin{verbatim}
## 
## Multiple Imputation using Bootstrap and PMM
## 
## aregImpute(formula = ~age + sex + slos + num.co + totcst + charges, 
##     data = support, n.impute = 10, type = "pmm")
## 
## n: 1000 	p: 6 	Imputations: 10  	nk: 3 
## 
## Number of NAs:
##     age     sex    slos  num.co  totcst charges 
##       0       0       0       0     105      25 
## 
##         type d.f.
## age        s    2
## sex        c    1
## slos       s    2
## num.co     s    2
## totcst     s    2
## charges    s    1
## 
## Transformation of Target Variables Forced to be Linear
## 
## R-squares for Predicting Non-Missing Values for Each Variable
## Using Last Imputations of Predictors
##  totcst charges 
##   0.908   0.910
\end{verbatim}
\end{kframe}
\end{knitrout}

The sample variance of multiple imputations is approximating the measurement
error that comes from the imputation itself. Since we are fitting an imputation
model to the data, our imputed values are also following the random error of
those fits; with multiple imputations, we can estimate that bias, which in turn
will yield estimates that are less biased that those from single imputations.
  
\bibliographystyle{vancouver}
\bibliography{result}
\end{document}
